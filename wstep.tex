\chapter{Wstęp}
\label{cha:wstep}

W ostatniej dekadzie można zauważyć zwiększone zainteresowanie rozwiązaniami z dziedziny
sztucznej inteligencji. Innowacyjne pomysły z użyciem tych algorytmów pozwalają nie tylko na interpretacje ogromnych 
ilości danych, których człowiek nie jest w stanie przetworzyć ale również, między innymi na rozwój 
autonomicznych pojazdów, jeżdżących bez nadzoru kierowcy.\\
Według \cite{mccarthy2007artificial} sztuczną inteligencją możemy nazwać ``badanie i rozwój inteligentnych maszyn, 
w szczególności programów komputerowych''.\\
Inteligentne zachowanie agenta możemy zdefiniować, gdy agent\cite{L.:2010:AIF:1809744}
\begin{itemize*}
 \item dostosowuje swoje zachowanie do aktualnych warunków i celów,
 \item ma zdolność zmiany otoczenia i celów,
 \item uczy się z doświadczenia,
 \item wykonuje odpowiednie do swoich ograniczeń akcję.
\end{itemize*}
Wykorzystując powyższe definicję, sztuczną inteligencje określamy jako dziedzinę naukową zajmującą się badaniem, 
rozwojem i implementacją programów i maszyn wykazujących cechy inteligencji, tzn. takie które ucząc się z 
doświadczenia, będąc zmiennym w stosunku do otaczającego ich otoczenia i warunków dążą do wykonania swoich celów 
uwzględniając obowiązujące je ograniczenia. 
%---------------------------------------------------------------------------

\section{Cele pracy}
\label{sec:celePracy}

Celem pracy jest opis i implementacja inteligentnego agenta. Agent wykorzystując algorytmy uczenia ze wzmacnianiem 
wyciąga wnioski z podejmowanych akcji i dostosowuje swoje zachowanie. W pracy zostanie wyjaśnione również inne rodzaje 
algorytmów uczenia ze wzmocnieniem.


%---------------------------------------------------------------------------

\section{Zawartość pracy}
\label{sec:zawartoscPracy}

Symulacja graficzna przedstawiająca agenta w środowisku.















