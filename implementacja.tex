\chapter{Implementacja}
\label{cha:implementacja}

Implementacja

\section{Opis implementacji}
\label{sec:opisimplementacji}

\subsection{Wybór algorytmów}
\label{subsec:wyboralgorytmow}

\subsubsection{Q-learning}
\label{subsubsec:qlearning}

\subsubsection{SARSA}
\label{subsubsec:sarsa}

\section{Wykorzystane technologie}
\label{sec:wykorzystanetechnologie}

Java 8 (środowisko IntelliJ Idea)
LibGdx (symulacja graficzna)
Project Lombok
Apache POI(czytanie i zapisywanie do xml)

\section{Symulacja graficzna}
\label{sec:symulacjagraficzna}

Proste prezentacja robota, środowiska
(przeszkoda, nagroda) w postaci figur
geometrycznych
Zamknięta przestrzeń dwuwymiarowa, po
której może się poruszać agent.

Akcje: MOVE_UP, MOVE_DOWN,
MOVE_RIGHT, MOVE_LEFT
● Reprezentacja stanu
siatka 3x3

\section{Napotkane problemy}
\label{sec:napotkaneproblemy}

Reprezentacja stanu
Testowanie działania algorytmów

\section{Wynik działania}
\label{sec:wynikdzialania}


%---------------------------------------------------------------------------














